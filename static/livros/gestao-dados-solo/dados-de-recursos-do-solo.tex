% Options for packages loaded elsewhere
\PassOptionsToPackage{unicode}{hyperref}
\PassOptionsToPackage{hyphens}{url}
%
\documentclass[
]{book}
\usepackage{lmodern}
\usepackage{amssymb,amsmath}
\usepackage{ifxetex,ifluatex}
\ifnum 0\ifxetex 1\fi\ifluatex 1\fi=0 % if pdftex
  \usepackage[T1]{fontenc}
  \usepackage[utf8]{inputenc}
  \usepackage{textcomp} % provide euro and other symbols
\else % if luatex or xetex
  \usepackage{unicode-math}
  \defaultfontfeatures{Scale=MatchLowercase}
  \defaultfontfeatures[\rmfamily]{Ligatures=TeX,Scale=1}
\fi
% Use upquote if available, for straight quotes in verbatim environments
\IfFileExists{upquote.sty}{\usepackage{upquote}}{}
\IfFileExists{microtype.sty}{% use microtype if available
  \usepackage[]{microtype}
  \UseMicrotypeSet[protrusion]{basicmath} % disable protrusion for tt fonts
}{}
\makeatletter
\@ifundefined{KOMAClassName}{% if non-KOMA class
  \IfFileExists{parskip.sty}{%
    \usepackage{parskip}
  }{% else
    \setlength{\parindent}{0pt}
    \setlength{\parskip}{6pt plus 2pt minus 1pt}}
}{% if KOMA class
  \KOMAoptions{parskip=half}}
\makeatother
\usepackage{xcolor}
\IfFileExists{xurl.sty}{\usepackage{xurl}}{} % add URL line breaks if available
\IfFileExists{bookmark.sty}{\usepackage{bookmark}}{\usepackage{hyperref}}
\hypersetup{
  pdftitle={Dados de Recursos do Solo},
  pdfauthor={Alessandro Samuel-Rosa},
  hidelinks,
  pdfcreator={LaTeX via pandoc}}
\urlstyle{same} % disable monospaced font for URLs
\usepackage{longtable,booktabs}
% Correct order of tables after \paragraph or \subparagraph
\usepackage{etoolbox}
\makeatletter
\patchcmd\longtable{\par}{\if@noskipsec\mbox{}\fi\par}{}{}
\makeatother
% Allow footnotes in longtable head/foot
\IfFileExists{footnotehyper.sty}{\usepackage{footnotehyper}}{\usepackage{footnote}}
\makesavenoteenv{longtable}
\usepackage{graphicx,grffile}
\makeatletter
\def\maxwidth{\ifdim\Gin@nat@width>\linewidth\linewidth\else\Gin@nat@width\fi}
\def\maxheight{\ifdim\Gin@nat@height>\textheight\textheight\else\Gin@nat@height\fi}
\makeatother
% Scale images if necessary, so that they will not overflow the page
% margins by default, and it is still possible to overwrite the defaults
% using explicit options in \includegraphics[width, height, ...]{}
\setkeys{Gin}{width=\maxwidth,height=\maxheight,keepaspectratio}
% Set default figure placement to htbp
\makeatletter
\def\fps@figure{htbp}
\makeatother
\setlength{\emergencystretch}{3em} % prevent overfull lines
\providecommand{\tightlist}{%
  \setlength{\itemsep}{0pt}\setlength{\parskip}{0pt}}
\setcounter{secnumdepth}{5}
\usepackage{booktabs}
\usepackage[]{natbib}
\bibliographystyle{apalike}

\title{Dados de Recursos do Solo}
\author{Alessandro Samuel-Rosa}
\date{2020-11-16}

\begin{document}
\maketitle

{
\setcounter{tocdepth}{1}
\tableofcontents
}
\hypertarget{prefuxe1cio}{%
\chapter*{Prefácio}\label{prefuxe1cio}}
\addcontentsline{toc}{chapter}{Prefácio}

\hypertarget{intro}{%
\chapter{Introdução}\label{intro}}

\hypertarget{variuxe1veis-morfoluxf3gicas}{%
\chapter*{Variáveis Morfológicas}\label{variuxe1veis-morfoluxf3gicas}}
\addcontentsline{toc}{chapter}{Variáveis Morfológicas}

\hypertarget{cor-do-solo}{%
\chapter{Cor do Solo}\label{cor-do-solo}}

\emph{Alessandro Samuel-Rosa}\footnote{Universidade Tecnológica Federal do Paraná, Curso de Agronomia, Prolongamento da Rua Cerejeira, s/n, CEP 85892-000, Santa Helena, Paraná, Brasil. E-mail: \href{mailto:alessandrorosa@utfpr.edu.br}{\nolinkurl{alessandrorosa@utfpr.edu.br}}.}

\hypertarget{conceitos-e-definiuxe7uxf5es}{%
\section{Conceitos e definições}\label{conceitos-e-definiuxe7uxf5es}}

A cor do solo é uma expressão qualitativa da percepção visual que seres humanos têm do solo. Logo, trata-se de uma qualidade do solo perceptível apenas na presença de fontes de luz visível, sejam elas artificiais ou naturais. Essa luz visível corresponde a comprimentos de onda do espectro eletromagnético entre cerca de 400 e 700 nm, porção popularmente conhecida como espectro óptico ou espectro do visível. Ignorados os principais fatores intervenientes (imperfeições do olho ou cérebro do observador, condições do ambiente circunvizinho, ângulo de incidência da luz e visada), a cor do solo resulta da repartição diferencial dos comprimentos de onda da luz visível incidente entre os fenômenos de absorção, transmissão e reflexão. Essa repartição diferencial é fruto da maneira como os vários materiais e substâncias que constituem o solo interagem com a radiação eletromagnética. São os comprimentos de onda do visível que deixam a superfície do solo em maior quantidade, por reflexão ou, inclusive, emissão, que ditam a cor do solo.

\hypertarget{fontes-adicionais-de-variauxe7uxe3o}{%
\section{Fontes adicionais de variação}\label{fontes-adicionais-de-variauxe7uxe3o}}

\hypertarget{padronizauxe7uxe3o}{%
\section{Padronização}\label{padronizauxe7uxe3o}}

O esquema padronizado de identificação e descrição de métodos de determinação da cor de materiais do solo utilizado no FEBR é apresentado na tabela abaixo. São considerados dois materiais de solo (matriz e manchas), dois estados de umidade (úmido e seco) e duas condições mecânicas (amassada e triturada).

\begin{longtable}[]{@{}ll@{}}
\toprule
\begin{minipage}[b]{0.51\columnwidth}\raggedright
Código de identificação\strut
\end{minipage} & \begin{minipage}[b]{0.43\columnwidth}\raggedright
Descrição mínima sugerida\strut
\end{minipage}\tabularnewline
\midrule
\endhead
\begin{minipage}[t]{0.51\columnwidth}\raggedright
\texttt{cor\_matriz\_umido\_munsell}\strut
\end{minipage} & \begin{minipage}[t]{0.43\columnwidth}\raggedright
Notação da cor de material de solo {[}\texttt{cor}{]}. Determinada na matriz do solo {[}\texttt{matriz}{]}. Amostra em estado de umidade úmido {[}\texttt{umido}{]}. Expressa utilizando a notação de Munsell® (matiz valor/croma) {[}\texttt{munsell}{]}. \{A DETERMINAÇÃO FOI FEITA NO CAMPO OU EM LABORATÓRIO?\}\strut
\end{minipage}\tabularnewline
\begin{minipage}[t]{0.51\columnwidth}\raggedright
\texttt{cor\_matriz\_seco\_munsell}\strut
\end{minipage} & \begin{minipage}[t]{0.43\columnwidth}\raggedright
Notação da cor de material de solo {[}\texttt{cor}{]}. Determinada na matriz do solo {[}\texttt{matriz}{]}. Amostra em estado de umidade seco {[}seco{]}. Expressa utilizando a notação de Munsell® (matiz valor/croma) {[}\texttt{munsell}{]}. \{ESPECIFICAR SE A DETERMINAÇÃO FOI FEITA NO CAMPO OU EM LABORATÓRIO\}\strut
\end{minipage}\tabularnewline
\begin{minipage}[t]{0.51\columnwidth}\raggedright
\texttt{cor\_matriz\_amassada\_munsell}\strut
\end{minipage} & \begin{minipage}[t]{0.43\columnwidth}\raggedright
Notação da cor de material de solo {[}\texttt{cor}{]}. Determinada na matriz do solo {[}\texttt{matriz}{]}. Amostra em estado de umidade úmido e condição mecânica amassada {[}\texttt{amassada}{]}. Expressa utilizando a notação de Munsell® (matiz valor/croma) {[}\texttt{munsell}{]}. \{ESPECIFICAR SE A DETERMINAÇÃO FOI FEITA NO CAMPO OU EM LABORATÓRIO\}\strut
\end{minipage}\tabularnewline
\begin{minipage}[t]{0.51\columnwidth}\raggedright
\texttt{cor\_matriz\_triturada\_munsell}\strut
\end{minipage} & \begin{minipage}[t]{0.43\columnwidth}\raggedright
Notação da cor de material de solo {[}\texttt{cor}{]}. Determinada na matriz do solo {[}\texttt{matriz}{]}. Amostra em estado de umidade seco e condição mecânica triturada {[}\texttt{triturada}{]}. Expressa utilizando a notação de Munsell® (matiz valor/croma) {[}\texttt{munsell}{]}. \{ESPECIFICAR SE A DETERMINAÇÃO FOI FEITA NO CAMPO OU EM LABORATÓRIO\}\strut
\end{minipage}\tabularnewline
\begin{minipage}[t]{0.51\columnwidth}\raggedright
\texttt{cor\_mancha\_umido\_munsell}\strut
\end{minipage} & \begin{minipage}[t]{0.43\columnwidth}\raggedright
Notação da cor de material de solo {[}\texttt{cor}{]}. Determinada na mancha do solo {[}\texttt{mancha}{]}. Amostra em estado de umidade úmido {[}\texttt{umido}{]}. Expressa utilizando a notação de Munsell® (matiz valor/croma) {[}\texttt{munsell}{]}. \{ESPECIFICAR SE A DETERMINAÇÃO FOI FEITA NO CAMPO OU EM LABORATÓRIO\}\strut
\end{minipage}\tabularnewline
\bottomrule
\end{longtable}

\hypertarget{exemplo}{%
\section{Exemplo}\label{exemplo}}

A tabela a seguir apresenta um exemplo de organização tabular de dados de cor da matriz e manchas de um perfil de solo. Os dados são reais e foram obtidos da descrição morfológica de um perfil de solo classificado como Planossolo Nátrico Órtico vertissólico, localizado no município de Pacaraima, Roraima\citep{OliveiraEtAl2018b}. A tabela está transposta, com as observações nas colunas e as variáveis nas linhas, para facilitar a visualização e comparação dos códigos de identificação da cores da matriz e machas do perfil de solo.

Dentre os oito horizontes, quatro possuem matriz com duas ou mais cores (Btgn, Btgnv1, Btgnv2 e 2Cgn). Nestes, as cores da matriz ocupam áreas superficiais aproximadamente similares, com distribuição espacial muito complexa, que dificulta a identificação visual de uma cor dominante com credibilidade. Trata-se do padrão multicolorido intricado conhecido como variegado.

\begin{longtable}[]{@{}lllllllll@{}}
\toprule
\begin{minipage}[b]{0.22\columnwidth}\raggedright
\strut
\end{minipage} & \begin{minipage}[b]{0.06\columnwidth}\raggedright
An\strut
\end{minipage} & \begin{minipage}[b]{0.06\columnwidth}\raggedright
En\strut
\end{minipage} & \begin{minipage}[b]{0.07\columnwidth}\raggedright
EBn\strut
\end{minipage} & \begin{minipage}[b]{0.07\columnwidth}\raggedright
Btn\strut
\end{minipage} & \begin{minipage}[b]{0.07\columnwidth}\raggedright
Btgn\strut
\end{minipage} & \begin{minipage}[b]{0.07\columnwidth}\raggedright
Btgnv1\strut
\end{minipage} & \begin{minipage}[b]{0.07\columnwidth}\raggedright
Btgnv2\strut
\end{minipage} & \begin{minipage}[b]{0.06\columnwidth}\raggedright
2Cgn\strut
\end{minipage}\tabularnewline
\midrule
\endhead
\begin{minipage}[t]{0.22\columnwidth}\raggedright
\texttt{cor\_matriz\_umido\_munsell\_1}\strut
\end{minipage} & \begin{minipage}[t]{0.06\columnwidth}\raggedright
10YR 5/1\strut
\end{minipage} & \begin{minipage}[t]{0.06\columnwidth}\raggedright
10YR 5/1\strut
\end{minipage} & \begin{minipage}[t]{0.07\columnwidth}\raggedright
10YR 6/2\strut
\end{minipage} & \begin{minipage}[t]{0.07\columnwidth}\raggedright
7,5YR 5/2\strut
\end{minipage} & \begin{minipage}[t]{0.07\columnwidth}\raggedright
7,5YR 6/2\strut
\end{minipage} & \begin{minipage}[t]{0.07\columnwidth}\raggedright
10YR 4/1\strut
\end{minipage} & \begin{minipage}[t]{0.07\columnwidth}\raggedright
10YR 4/1\strut
\end{minipage} & \begin{minipage}[t]{0.06\columnwidth}\raggedright
10YR 6/2\strut
\end{minipage}\tabularnewline
\begin{minipage}[t]{0.22\columnwidth}\raggedright
\texttt{cor\_matriz\_umido\_munsell\_2}\strut
\end{minipage} & \begin{minipage}[t]{0.06\columnwidth}\raggedright
-\strut
\end{minipage} & \begin{minipage}[t]{0.06\columnwidth}\raggedright
-\strut
\end{minipage} & \begin{minipage}[t]{0.07\columnwidth}\raggedright
-\strut
\end{minipage} & \begin{minipage}[t]{0.07\columnwidth}\raggedright
-\strut
\end{minipage} & \begin{minipage}[t]{0.07\columnwidth}\raggedright
7,5YR 5/2\strut
\end{minipage} & \begin{minipage}[t]{0.07\columnwidth}\raggedright
10YR 5/2\strut
\end{minipage} & \begin{minipage}[t]{0.07\columnwidth}\raggedright
10YR 5/2\strut
\end{minipage} & \begin{minipage}[t]{0.06\columnwidth}\raggedright
10YR 7/2\strut
\end{minipage}\tabularnewline
\begin{minipage}[t]{0.22\columnwidth}\raggedright
\texttt{cor\_matriz\_umido\_munsell\_3}\strut
\end{minipage} & \begin{minipage}[t]{0.06\columnwidth}\raggedright
-\strut
\end{minipage} & \begin{minipage}[t]{0.06\columnwidth}\raggedright
-\strut
\end{minipage} & \begin{minipage}[t]{0.07\columnwidth}\raggedright
-\strut
\end{minipage} & \begin{minipage}[t]{0.07\columnwidth}\raggedright
-\strut
\end{minipage} & \begin{minipage}[t]{0.07\columnwidth}\raggedright
-\strut
\end{minipage} & \begin{minipage}[t]{0.07\columnwidth}\raggedright
10YR 6/2\strut
\end{minipage} & \begin{minipage}[t]{0.07\columnwidth}\raggedright
10YR 4/2\strut
\end{minipage} & \begin{minipage}[t]{0.06\columnwidth}\raggedright
-\strut
\end{minipage}\tabularnewline
\begin{minipage}[t]{0.22\columnwidth}\raggedright
\texttt{cor\_matriz\_umido\_munsell\_4}\strut
\end{minipage} & \begin{minipage}[t]{0.06\columnwidth}\raggedright
-\strut
\end{minipage} & \begin{minipage}[t]{0.06\columnwidth}\raggedright
-\strut
\end{minipage} & \begin{minipage}[t]{0.07\columnwidth}\raggedright
-\strut
\end{minipage} & \begin{minipage}[t]{0.07\columnwidth}\raggedright
-\strut
\end{minipage} & \begin{minipage}[t]{0.07\columnwidth}\raggedright
-\strut
\end{minipage} & \begin{minipage}[t]{0.07\columnwidth}\raggedright
7,5YR 5/8\strut
\end{minipage} & \begin{minipage}[t]{0.07\columnwidth}\raggedright
7,5YR 5/6\strut
\end{minipage} & \begin{minipage}[t]{0.06\columnwidth}\raggedright
-\strut
\end{minipage}\tabularnewline
\begin{minipage}[t]{0.22\columnwidth}\raggedright
\texttt{cor\_matriz\_seco\_munsell}\strut
\end{minipage} & \begin{minipage}[t]{0.06\columnwidth}\raggedright
10YR 6/2\strut
\end{minipage} & \begin{minipage}[t]{0.06\columnwidth}\raggedright
10YR 6/1\strut
\end{minipage} & \begin{minipage}[t]{0.07\columnwidth}\raggedright
-\strut
\end{minipage} & \begin{minipage}[t]{0.07\columnwidth}\raggedright
-\strut
\end{minipage} & \begin{minipage}[t]{0.07\columnwidth}\raggedright
-\strut
\end{minipage} & \begin{minipage}[t]{0.07\columnwidth}\raggedright
-\strut
\end{minipage} & \begin{minipage}[t]{0.07\columnwidth}\raggedright
-\strut
\end{minipage} & \begin{minipage}[t]{0.06\columnwidth}\raggedright
-\strut
\end{minipage}\tabularnewline
\begin{minipage}[t]{0.22\columnwidth}\raggedright
\texttt{cor\_mancha\_umido\_munsell\_1}\strut
\end{minipage} & \begin{minipage}[t]{0.06\columnwidth}\raggedright
-\strut
\end{minipage} & \begin{minipage}[t]{0.06\columnwidth}\raggedright
-\strut
\end{minipage} & \begin{minipage}[t]{0.07\columnwidth}\raggedright
7,5YR 5/8\strut
\end{minipage} & \begin{minipage}[t]{0.07\columnwidth}\raggedright
7,5YR 5/8\strut
\end{minipage} & \begin{minipage}[t]{0.07\columnwidth}\raggedright
-\strut
\end{minipage} & \begin{minipage}[t]{0.07\columnwidth}\raggedright
-\strut
\end{minipage} & \begin{minipage}[t]{0.07\columnwidth}\raggedright
-\strut
\end{minipage} & \begin{minipage}[t]{0.06\columnwidth}\raggedright
10YR 6/8\strut
\end{minipage}\tabularnewline
\begin{minipage}[t]{0.22\columnwidth}\raggedright
\texttt{cor\_mancha\_umido\_munsell\_2}\strut
\end{minipage} & \begin{minipage}[t]{0.06\columnwidth}\raggedright
-\strut
\end{minipage} & \begin{minipage}[t]{0.06\columnwidth}\raggedright
-\strut
\end{minipage} & \begin{minipage}[t]{0.07\columnwidth}\raggedright
-\strut
\end{minipage} & \begin{minipage}[t]{0.07\columnwidth}\raggedright
7,5YR 4/1\strut
\end{minipage} & \begin{minipage}[t]{0.07\columnwidth}\raggedright
-\strut
\end{minipage} & \begin{minipage}[t]{0.07\columnwidth}\raggedright
-\strut
\end{minipage} & \begin{minipage}[t]{0.07\columnwidth}\raggedright
-\strut
\end{minipage} & \begin{minipage}[t]{0.06\columnwidth}\raggedright
-\strut
\end{minipage}\tabularnewline
\begin{minipage}[t]{0.22\columnwidth}\raggedright
\texttt{cor\_mancha\_umido\_munsell\_3}\strut
\end{minipage} & \begin{minipage}[t]{0.06\columnwidth}\raggedright
-\strut
\end{minipage} & \begin{minipage}[t]{0.06\columnwidth}\raggedright
-\strut
\end{minipage} & \begin{minipage}[t]{0.07\columnwidth}\raggedright
-\strut
\end{minipage} & \begin{minipage}[t]{0.07\columnwidth}\raggedright
7,5YR 7/2\strut
\end{minipage} & \begin{minipage}[t]{0.07\columnwidth}\raggedright
-\strut
\end{minipage} & \begin{minipage}[t]{0.07\columnwidth}\raggedright
-\strut
\end{minipage} & \begin{minipage}[t]{0.07\columnwidth}\raggedright
-\strut
\end{minipage} & \begin{minipage}[t]{0.06\columnwidth}\raggedright
-\strut
\end{minipage}\tabularnewline
\bottomrule
\end{longtable}

\hypertarget{variuxe1veis-fuxedsicas}{%
\chapter*{Variáveis Físicas}\label{variuxe1veis-fuxedsicas}}
\addcontentsline{toc}{chapter}{Variáveis Físicas}

\hypertarget{densidade-do-solo-inteiro}{%
\chapter{Densidade do Solo Inteiro}\label{densidade-do-solo-inteiro}}

\emph{Alessandro Samuel-Rosa}\footnote{Universidade Tecnológica Federal do Paraná, Curso de Agronomia, Prolongamento da Rua Cerejeira, s/n, CEP 85892-000, Santa Helena, Paraná, Brasil. E-mail: \href{mailto:alessandrorosa@utfpr.edu.br}{\nolinkurl{alessandrorosa@utfpr.edu.br}}.}, \emph{Wenceslau Geraldes Teixeira}\footnote{Empresa Brasileira de Pesquisa Agropecuária, Centro Nacional de Pesquisa de Solos, Rua Jardim Botânico, 1024, CEP 22460-000, Rio de Janeiro, Rio de Janeiro, Brasil. E-mail: \href{mailto:wenceslau.teixeira@embrapa.br}{\nolinkurl{wenceslau.teixeira@embrapa.br}}.}, \emph{João Herbert Moreira Viana}\footnote{Empresa Brasileira de Pesquisa Agropecuária, Centro Nacional de Pesquisa de Milho e Sorgo, Rodovia MG-424, Km 45, CEP 35701-970, Sete Lagoas, Minas Gerais, Brasil. E-mail: \href{mailto:joao.herbert@embrapa.br}{\nolinkurl{joao.herbert@embrapa.br}}.}

\hypertarget{conceitos-e-definiuxe7uxf5es-1}{%
\section{Conceitos e definições}\label{conceitos-e-definiuxe7uxf5es-1}}

A densidade do solo, também chamada erroneamente de densidade aparente ou densidade global, é uma medida da massa seca de todo o material contido em determinado volume de solo. Esse material inclui todas as frações dos componentes de origem mineral e orgânica do solo. No FEBR, a densidade do solo é definida, operacionalmente, como densidade do solo inteiro (DSI), e seu código de identificação é \texttt{dsi}. No World Soil Information Service (WoSIS), a densidade do solo inteiro é denominada \emph{bulk density whole soil}, e seu código de identificação é BDWS.

\hypertarget{muxe9todos-de-determinauxe7uxe3o}{%
\section{Métodos de determinação}\label{muxe9todos-de-determinauxe7uxe3o}}

A DSI pode ser determinada utilizando uma variedade de métodos diretos e indiretos. Aqui tratamos apenas dos métodos diretos. Esses métodos diferenciam-se em si pela maneira de obtenção das amostras do solo e quantificação (ou estimativa) da sua massa e volume. Em geral, a escolha por um ou outro método está relacionada às características do solo, especialmente a granulometria e composição mineralógica do solo. Os principais métodos diretos de determinação da DSI são: cilindro, torrão, proveta, monolito, e escavação. Assim, o segundo nível de codificação da DSI no FEBR consiste na identificação desses métodos, ou seja:

\begin{itemize}
\tightlist
\item
  \texttt{dsi\_cilindro},
\item
  \texttt{dsi\_torrao},
\item
  \texttt{dsi\_proveta},
\item
  \texttt{dsi\_monolito}, e
\item
  \texttt{dsi\_escavacao}.
\end{itemize}

Dentre os métodos diretos, \texttt{dsi\_cilindro} e \texttt{dsi\_torrao} são os mais utilizados quando o material do solo é coletado tentando se preservar a estrutura do solo. Também costumam ser os mais utilizados quando o solo é composto, em sua quase totalidade, pela fração fina (\textgreater{} 2 mm) e, no caso de \texttt{dsi\_torrao}, quando há predomínio da fração argila. Já o método \texttt{dsi\_proveta} aplica-se mais aos casos em que há dominância da fração areia ou quando se pretende fazer estimativas da DSI usando apenas a fração fina do solo. Por fim, os métodos \texttt{dsi\_monolito} e \texttt{dsi\_escavacao} são mais usados quando há predomínio das frações grossas (\textgreater{} 2 mm), que impossibilitam ou dificultam sobremaneira a inserção de cilindros no solo ou coleta de torrões com estrutura preservada. As mesmas dificuldades são encontradas em solo com expressivo volume de raízes de grande diâmetro e elevado conteúdo de matéria orgânica, como em florestas, onde o método \texttt{dsi\_escavacao} é o mais usado.

\hypertarget{fontes-adicionais-de-variauxe7uxe3o-1}{%
\section{Fontes adicionais de variação}\label{fontes-adicionais-de-variauxe7uxe3o-1}}

Cada um dos métodos diretos de determinação da DSI apontados acima abrange uma variedade de detalhes metodológicos que podem exercer pequenas influências sobre os resultados obtidos. Por exemplo, no caso do \texttt{dsi\_cilindro}, há chance de influência do método utilizado para inserção do cilindro no solo, que pode ser por percussão com martelo ou por pressão com prensa hidráulica. Em geral, espera-se que o método da percussão com martelo cause maior compactação e estilhaçamento da amostra, especialmente quando o cilindro usado é muito pequeno \citep{CasanovaEtAl2016}. A compactação da amostra também pode ocorrer quando os cilindros usados não possuem corte em bisel na face inserida no solo. Por outro lado, espera-se que, quanto mais finas as paredes do cilindro, menor seja a compactação da amostra.

Outro detalhe metodológico que pode influenciar os resultados é o tamanho do cilindro \citep{Al-ShammaryEtAl2018, CasanovaEtAl2016}. Boa parte dos estudos usa cilindros de 100 cm3, tamanho recomendado pelo Manual de Métodos de Análise de Solo \citep{AlmeidaEtAl2017}. Contudo, relatos do uso de cilindros de diversos tamanhos são encontrados na literatura, por exemplo, 9,65 cm3 (amostras pequenas usadas em reometria), 30 cm3, 300 cm3, e 1000 cm3 (calibração de sensores de medição do conteúdo de água no solo).

A definição do tamanho do cilindro é importante pois está relacionada ao volume elementar representativo (REV, do inglês \emph{representative elementar volume}) de cada tipo de solo. Por exemplo, se há poros muito grandes ou cascalhos e calhaus no sistema, cilindros pequenos não seriam representativos. A questão do REV se aplica também aos demais métodos, principalmente \texttt{dsi\_torrao}, \texttt{dsi\_monolito} e \texttt{dsi\_escavacao}. Especialmente no caso do \texttt{dsi\_monolito} e \texttt{dsi\_escavacao}, que utilizam amostras maiores, a questão da REV seria melhor atendida. Esse já não seria o caso do método \texttt{dsi\_torrao}, onde torrões de tamanhos bastantes variados são usados, mas geralmente menores que 100 m³ devido à dificuldade de se obter agregados estáveis -- especialmente em solo de granulometria mais grossa. Um aspecto importante a ser considerado aqui é que o uso de repetições locais não resolve o problema da REV. Isso porque a função das repetições é quantificar o erro de medida, não resolver variação espacial de curtíssima distância -- no FEBR, repetições são identificada na tabela \emph{camada} usando o campo \texttt{amostra\_id}.

Diferentes dos demais métodos, \texttt{dsi\_torrao} e \texttt{dsi\_monolito} incluem a impermeabilização das amostras. Esse procedimento pode ser feito usando diferentes compostos, tais como parafina, verniz ou querosene. Apesar das diferenças entre esses compostos, espera-se que o uso de diferentes impermeabilizantes tenha efeito muitíssimo pequeno, se algum, sobre os resultados de DSI obtidos. O mesmo pode ser esperado para o caso do \texttt{dsi\_escavacao}, método em que podem ser utilizados diferentes materiais para preenchimento da escavação, por exemplo, areia ou água.

Um detalhe metodológico de grande importância na determinação da DSI é a umidade da amostra do solo no momento da quantificação da sua massa e volume. Na maioria dos casos, esses valores são obtidos com a amostra seca. Contudo, para as amostras de alguns tipos de solo, o que se faz é equilibrar as amostras em determinado potencial, geralmente 1/3 atm ou 33 kPa, que corresponde ao que se convencionou chamar de ponto de murcha permanente em muitos países. Esse procedimento se justifica nos casos de solos com presença de argilas expansivas em quantidade suficiente para causar aumento perceptível do volume da massa de solo, como no caso dos solos com características vérticas e Vertissolos. Na determinação da DSI desses solos, o volume e massa da amostra são quantificados nesta umidade controlada \citep{MathieuEtAl1998}.

Dentre todos os detalhes metodológicos mencionados acima, a condição de umidade da amostra de solo no momento da quantificação da massa e volume é aquele com maior potencial de exercer influências sobre os resultados obtidos. Para a maioria das outras fontes adicionais de variação nos resultados, os trabalhos de comparação, quando existentes, são de difícil acesso. No FEBR, ainda não foram registrados conjuntos de dados em que a determinação da DSI tenha sido realizada com amostras úmidas, razão pela qual não se faz necessário definir codificação adicional. Contudo, caso dados dessa natureza sejam submetidos ao FEBR, a codificação poderá ser definida, em seu terceiro nível, pela condição da amostra no momento da quantificação da massa e volume -- por exemplo, \texttt{estufa}, \texttt{ar}, \texttt{campo}, \texttt{33kpa}, entre outras. Esse procedimento estaria em acordo com aqueles adotados no World Soil Information System (WoSIS) \citep{RibeiroEtAl2018}.

\hypertarget{padronizauxe7uxe3o-1}{%
\section{Padronização}\label{padronizauxe7uxe3o-1}}

A Tabela 1 apresenta os códigos padronizados utilizados no FEBR para identificar cada um dos métodos de determinação da DSI. Também apresenta uma descrição mínima sugerida para cada um desses métodos, ou seja, um roteiro padronizado para especificação dos detalhes mais importantes de cada método. Detalhes adicionais dos métodos podem ser especificados conforme indicado pelo texto em caixa alta entre chaves.

\textbf{Tabela 1.} Codificação e descrição mínima sugerida da densidade do solo inteiro no FEBR.

Campo

Descrição mínima sugerida

dsi\_cilindro

Densidade do solo inteiro {[}dsi{]}. Coleta de amostra do solo com estrutura preservada usando cilindro de volume interno conhecido, com determinação da massa da amostra por pesagem {[}cilindro{]}. \{DESCREVER CILINDRO: DIMENSÕES, FORMA DA FACE\} \{DESCREVER MÉTODO DE INSERÇÃO DO CILINDRO: PERCUSSÃO, MACACO HIDRÁULICO\} \{ESPECIFICAR CONDIÇÃO DE UMIDADE DA AMOSTRA DURANTE DETERMINAÇÃO DA MASSA\} \{INDICAR REFERÊNCIA DO MÉTODO\}

dsi\_torrao

Densidade do solo inteiro {[}dsi{]}. Coleta de amostra do solo (torrão) com estrutura preservada, posteriormente impermeabilizada para determinação do volume pelo deslocamento de líquido, com determinação da massa da amostra por pesagem {[}torrao{]}. \{DESCREVER TORRÃO: DIMENSÕES, FORMA\} \{ESPECIFICAR COMPOSTO IMPERMEABILIZANTE USADO\} \{ESPECIFICAR CONDIÇÃO DE UMIDADE DA AMOSTRA DURANTE DETERMINAÇÃO DA MASSA\} \{INDICAR REFERÊNCIA DO MÉTODO\}

dsi\_proveta

Densidade do solo inteiro {[}dsi{]}. Obtenção da massa por pesagem após compactação da amostra de solo em uma proveta até o volume pré-determinado {[}proveta{]}. \{ESPECIFICAR AS CONDIÇÕES DE COMPACTAÇÃO DA AMOSTRA: MANUAL, VIBRAÇÃO MECÂNICA\} \{INDICAR REFERÊNCIA DO MÉTODO\}

dsi\_monolito

Densidade do solo inteiro {[}dsi{]}. Coleta de amostra do solo (monolito) com estrutura preservada, posteriormente impermeabilizada com resina ou verniz para determinação do volume pelo deslocamento de líquido, com determinação da massa da amostra por pesagem {[}monolito{]}. \{DESCREVER MONOLITO: DIMENSÕES, FORMA\} \{ESPECIFICAR CONDIÇÃO DE UMIDADE DA AMOSTRA DURANTE DETERMINAÇÃO DA MASSA\} \{INDICAR REFERÊNCIA DO MÉTODO\}

dsi\_escavacao

Densidade do solo inteiro {[}dsi{]}. Determinação da massa por pesagem do material da escavação e do volume deixado no solo escavado por meio do preenchimento do espaço vazio por meio de um fluido ou outro material que preenche o espaço {[}escavacao{]}. \{DESCREVER ESCAVAÇÃO: DIMENSÕES, FORMA\} \{ESPECIFICAR CONDIÇÃO DE UMIDADE DA AMOSTRA DURANTE DETERMINAÇÃO DA MASSA\} \{ESPECIFICAR O FLUIDO OU MATERIAL/MÉTODO USADO\} \{INDICAR REFERÊNCIA DO MÉTODO\}

dsi\_xxx

Densidade do solo inteiro {[}dsi{]}. Coleta de amostra do solo e determinação da massa e volume da amostra do solo usando método não especificado {[}xxx{]}.

A Tabela 2 abaixo apresenta as especificações de unidade de medida, a precisão numérica, tipo de dado, e categoria da variável no FEBR.

\textbf{Tabela 2.} Especificações da densidade do solo inteiro no FEBR.

Unidade de medida

Precisão numérica

Tipo de dado

Categoria da variável

kg/dm\^{}3

2

Real

Física

\hypertarget{harmonizauxe7uxe3o}{%
\section{Harmonização}\label{harmonizauxe7uxe3o}}

Existe uma grande variedade de métodos de determinação direta da DSI. Cada um deles foi desenvolvido de maneira a permitir a determinação da DSI em diferentes condições, por exemplo, de granulometria e mineralogia do solo. Assim, não existe um único método de referência para a determinação da DSI. Isso significa que, em geral, não é possível transformar os valores obtidos com um método qualquer para valores aproximadamente equivalentes aos valores que seriam obtidos com outro método. Dados empíricos que permitam construir modelos estatísticos para fazer essa conversão são escassos e pouco representativos.

A limitação de dados empíricos impossibilita, por ora, a harmonização dos dados de DSI no FEBR. Estudos que venham resolver essa limitação são bem vindos, especialmente no caso dos métodos que guardam mais similaridades entre si, por exemplo, \texttt{dsi\_torrao} e \texttt{dsi\_cilindro}. Em geral, \texttt{dsi\_torrao} tende a sobre estimar a DSI, comparado ao \texttt{dsi\_cilindro}, por desconsiderar o espaço poroso existente entre os agregados \citep{CasanovaEtAl2016}. O FEBR tem condições de acomodar um modelo estatístico que transforme os valores de DSI obtidos com \texttt{dsi\_torrao} para valores aproximadamente equivalentes àqueles que seriam obtidos com \texttt{dsi\_cilindro} -- e vice-versa.

\hypertarget{agradecimentos}{%
\section*{Agradecimentos}\label{agradecimentos}}
\addcontentsline{toc}{section}{Agradecimentos}

\begin{itemize}
\tightlist
\item
  Iêde de Brito Chaves, professor aposentado da Universidade Federal da Paraíba (UFPB), pelas considerações acerca das potenciais dificuldades causadas pelo uso das expressões `densidade do solo inteiro', `densidade do solo', `densidade aparente' e `densidade global'. {[}\href{https://groups.google.com/d/msgid/soil-mapping/7c8745ca-243d-4d15-97cc-2d00cb5eefdd\%40googlegroups.com}{Ver essa discussão na Web}{]}
\item
  Anderson Sandro da Rocha, professor da Universidade Tecnológica Federal do Paraná, Curso de Agronomia.
\item
  Paulo Ivonir Gubiani, professor da Universidade Federal de Santa Maria, Departamento de Solos.
\end{itemize}

  \bibliography{book.bib,packages.bib}

\end{document}
